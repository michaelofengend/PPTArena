\section{An Effective PPT Editing Agent: PPTPilot}
\label{sec:pptpilot}

We introduce PPTPilot, an agent for presentation editing, whose simple architecture yields surprisingly effective results, even outperforming proprietary products like OpenAI's ChatGPT Agent (details in Sec.~\ref{sec:main_comparison}). Our design is built on two key insights. \textbf{(1)} The primary challenge in this domain is not solely intent recognition, but reliability and precision. PowerPoint files are built on the brittle Office Open XML (OOXML) format, which is highly intolerant to malformed or ``hallucinated'' VLM outputs, and therefore requires specialized formats and contexts suitable for PPT editing. \textbf{(2)} We found that no single editing modality, \eg, purely relying on the XML format, is sufficient. A robust agent must be capable of intelligently selecting the optimal tool and editing interface for a given task.

\begin{figure}[t]
    \centering
    \includegraphics[width=1.00\linewidth]{figures/PPTPilotflow.pdf}
    \caption{\textbf{Our PPTPilot paradigm}. Our key insight highlights a combination of two different editing skills: functional code and direct XML edits to control the fine-grained structural elements. A ``Skill Router'' determines which skill is more suitable for a query. And then the corresponding VLM executes the edits via either route. Notably, the editing operations can also be enhanced with reflection, trading time for more reliable editing. For the router we use a fast LLM (GPT-5 nano or Gemini-2.5 flash), and then GPT-5 for our VLM edit calls.}
    \vspace{-3mm}
    \label{fig:pptpilot_diagram}
\end{figure}


PPTPilot's core design is a dual-path architecture, emphasizing the ability to handle editing queries via either programmatic tools or direct XML editing (Fig.~\ref{fig:pptpilot_diagram}). This hybrid design enables our agent to address a wide range of editing queries reliably and in a principled way.

\mypar{Programmatic Editing.}
Utilizing \texttt{python-pptx} to edit PPTs programmatically (a method commonly adopted in prior work \cite{jung2025talkslideslanguagedrivenagents,pptagent2025,ge2025autopresentdesigningstructuredvisuals,autoslides2025,pang2025paper2postermultimodalposterautomation} scripts the edits by generating code (top of Fig.~\ref{fig:pptpilot_diagram}). This approach is highly effective for repetitive, well-defined, and content-centric operations, such as performing a ``find-and-replace'' across all slides or translating text. However, it lacks the fine-grained control required for complex structural modifications (\emph{\eg}, altering slide masters, themes, or specific layout geometries).

\mypar{Direct XML Editing.}
To address the limitations of the programmatic path in structural and visual editing scenarios, we have equipped PPTPilot with a second skill: the ability to directly read, parse, and re-write raw OOXML files (\eg, \texttt{slide1.xml}, \texttt{theme.xml}), as shown in the top half of Fig.~\ref{fig:pptpilot_diagram}. This approach provides the precision required for structured contexts, as the VLMs can directly manipulate fine-grained properties like the specific positions of elements. Since OOXML encodes most of the information in a PPT, the XML path provides a unified interface well-aligned with existing VLMs for PPT editing. However, the long context and strict format requirements of XML make it challenging to perform precise edits, especially when modifications span a large number of slides, in which case the programmatic approach is significantly more reliable.

\mypar{Skill Routing.} To determine which editing skills to adopt for a specific user query, we employ a VLM that routes the query to the proper editing skills, as the beginning of branching in Fig.~\ref{fig:pptpilot_diagram}. Upon receiving a user instruction, this decider analyzes the prompt combined withthe presentation's structure, including the screenshots and contents. Based on this analysis, it routes the task to either the programmatic path or the direct XML editing path.

\mypar{Self-correction with Reflection.} Finally, we acknowledge the complexity of PPT editing, which indicates the challenge of correct edits in a single try. Inspired by representative agents like ReACT~\cite{yao2022react}, we introduce an iterative reflection path into the PPTPilot, so that it can gradually refine its predictions. The agent proposes an edit to the XML files, which is rendered to temporarily to a PPT file. Then a verifier model assesses the output PPT according to the original instructions and provides feedbacks for failures. In this way, the agent produces an updated PPT edit based on the feedback and is able to correct its own errors.

Despite the simplicity of our design, we find it effective and efficient for PPT editing when compared against existing agent products and frameworks. We hope our PPTPilot can serve as a baseline for research into PPT editing.
