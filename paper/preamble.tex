%% This file contains a number of tweaks that are typically applied to the main document.
%% They are not enabled by default, but can be enabled by uncommenting the relevant lines.

%%
%% Inline annotations; for predefined colors, refer to "dvipsnames" in the xcolor package:
%% https://tinyurl.com/overleaf-colors
%%
%% disable for camera ready / submission by uncommenting these lines  
%%
% \renewcommand{\TODO}[1]{}
% \renewcommand{\todo}[1]{#1}

%% Figures and plots
\usepackage{amssymb}
\usepackage{etoolbox}
\usepackage{capt-of}
\usepackage{array}
\usepackage{multirow}
\usepackage{tikz}
\usetikzlibrary{arrows.meta,positioning,calc,decorations.pathreplacing,shapes.misc,fit,spy}
\usepackage{pgfplots}
\pgfplotsset{compat=1.18}
\usepackage{stfloats}
\usepackage{float}
\usepackage{lineno}
\usepackage{placeins}
\usepackage{colortbl}
\graphicspath{{PPTPilotimages/}}
\newcommand{\rot}[1]{\rotatebox[origin=c]{90}{#1}}
\usepackage{xcolor}
\newcommand{\placeholder}{\textcolor{gray}{--}}

\usepackage{etoolbox}
\usepackage{listings}

\definecolor{darkback}{RGB}{20,20,20}
\definecolor{codecyan}{RGB}{86,156,214}
\definecolor{codelime}{RGB}{106,153,85}
\definecolor{codeorange}{RGB}{206,145,120}
\definecolor{codepurple}{RGB}{197,134,192}
\definecolor{codeblue}{RGB}{86,156,214}
\definecolor{codewhite}{RGB}{220,220,220}

\lstdefinestyle{nightmode}{
    backgroundcolor=\color{darkback},
    commentstyle=\color{codelime},
    keywordstyle=\color{codecyan},
    numberstyle=\tiny\color{gray},
    stringstyle=\color{codeorange},
    basicstyle=\ttfamily\scriptsize\color{codewhite},
    breakatwhitespace=false,
    breaklines=true,
    captionpos=b,
    keepspaces=true,
    numbers=left,
    numbersep=15pt,
    showspaces=false,
    showstringspaces=false,
    showtabs=false,
    tabsize=2,
    frame=none,
    columns=fixed,
    xleftmargin=30pt,
    framexleftmargin=5pt,
    lineskip=-0.5pt,
    rulesep=0pt
}

\lstset{
    style=nightmode
}

\makeatletter
% Turn OFF writing entries to the .toc
\newcommand{\suppressmaintoc}{%
  \let\origaddcontentsline\addcontentsline
  \renewcommand{\addcontentsline}[3]{}%
}

\newcommand\blfootnote[1]{%
  \begingroup
  \renewcommand\thefootnote{}\footnote{#1}%
  \addtocounter{footnote}{-1}%
  \endgroup
}

% Turn ON writing entries to the .toc
\newcommand{\restoretoc}{%
  \let\addcontentsline\origaddcontentsline
}
\makeatother


%%
%% work harder in optimizing text layout. Typically shrinks text by 1/6 of page, enable
%% it at the very end of the writing process, when you are just above the page limit
%%
% \usepackage{microtype}

%%
%% fine-tune paragraph spacing
%%
% \renewcommand{\paragraph}[1]{\vspace{.5em}\noindent\textbf{#1.}}
%%
%% globally adjusts space between figure and caption
%%
% \setlength{\abovecaptionskip}{.5em}


%%
%% Allows "the use of \paper to refer to the project name"
%% with automatic management of space at the end of the word
%%
% \usepackage{xspace}
% \newcommand{\paper}{ProjectName\xspace}

%%
%% Commonly used math definitions
%%
% \DeclareMathOperator*{\argmin}{arg\,min}
% \DeclareMathOperator*{\argmax}{arg\,max}

%%
%% Tigthen underline
%%
% \usepackage{soul}
% \setuldepth{foobar}
